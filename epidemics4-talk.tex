\documentclass{article}

\usepackage{amsmath,hyperref}

% \newcommand{\Mp}{\ensuremath{{M_{+}}}}
% \newcommand{\barMp}{\ensuremath{\overline{\Mp}}}
% \newcommand{\Dt}{\ensuremath{\Delta t}}
% \newcommand{\tM}[1]{\ensuremath{\tilde{M}(#1)}}
% \newcommand{\tMt}{\tM{\tau}}
% \newcommand{\tMB}{\ensuremath{{\tilde{M}_B}}}
% \newcommand{\tE}[1]{\ensuremath{{\tilde{E}(#1)}}}
% \newcommand{\tEt}{\tE{\tau}}

\author{Carl~A.~B.~Pearson, Tom~J.~Hladish}
%\address{Emerging Pathogens Institute, University of Florida, Gainesville, FL}

\usepackage{Sweave}
\begin{document}
\input{epidemics4-talk-concordance}
\title{Identifying Disease Source on a Network with Limited Information}

%\institute{Emerging Pathogens Institute, University of Florida}
\maketitle
\begin{centering}
\section*{Abstract}
\end{centering}
We compare the relative success of multiple approaches to identifying the source of a simulated series of outbreaks on a real-world network when information is severely limited.

The performance of these approaches is limited by several constraints that can appear in real-world situations: the structure of the network is not directly available, the disease parameters are unknown {\em a priori}, reporting of the disease is limited -- {\em e.g.}, because cases are misidentified, public health reporting is limited, or the disease is typically asymptomatic -- and active investigation results are highly time-sensitive.


% \section{A SIMPLE LIST OF POINTS}
% \begin{itemize}
% \item point one,
% \item two,
% \item see
% \end{itemize}
% 
% \section{INSERTING AN R-GENERATED FIGURE}
% \begin{figure}
% \begin{center}
%<<plotfig1,fig=TRUE,echo=FALSE>>=
% periodT <- 366; omega <- 2*pi/periodT
% monthDays<-c(31,29,31,30,31,30,31,31,30,31,30,31)
% monthOffset<- -(rev(cumsum(monthDays)) + -periodT/2)
% monthInset<- monthOffset + 15
% fun <- function(x) { cos(omega*x) }
% cplot <- function() {
% plot(fun, xlim=c(-periodT/2,periodT/2), 
%      xlab="",yaxt="n",xaxt="n",ylab="Mosquito Abundance",type="l",col="blue")
% axis(side=1, at=c(monthOffset,periodT/2), labels=F)
% axis(side=1, at=monthInset, labels=c("JAN","FEB","MAR","APR","MAY","JUN","JUL","AUG","SEP","OCT","NOV","DEC"),tick=F, cex.axis=0.7)
% }
% cplot()
%@
% \end{center}
% \end{figure}
% notes-notes-notes

% \section{INSERT ANOTHER PDF}
% \begin{figure}
% \begin{center}
% \includegraphics[width=0.9\textwidth]{insert.pdf}
% \caption{Bicout et al. J. Med. Entomol. 43(5): 936-946 (2006)}
% \end{center}
% \end{figure}
% this is a not uncommon measurement of a seasonally varying vector population.

% \section{SHOW SOME MATH}
% $$
% M(t) = C\sin(\omega t+\theta)
% $$
% 
% \section{SEVERAL EQUATIONS}
% \begin{align*}
% E(t) &= \begin{cases}
%         \dfrac{\Mp}{\Dt} & t \in \Dt \\
%         0 & \textrm{otherwise}
%         \end{cases}\tag{Step} \\
% E(\rho, t) &= \begin{cases}
%         \dfrac{2\Mp}{\Dt(2-\rho)} & t \in \Dt(1-\rho) \\
%         \dfrac{2\Mp}{\Dt(2-\rho)\rho}\left(1-\dfrac{2|t|}{\Dt}\right) & t \in \rho\Dt \\
%         0 & \textrm{otherwise}
%         \end{cases}\tag{Modified Step} \\
% E(t) &= \dfrac{2\Mp}{\Dt}\sqrt{\dfrac{2}{\pi}}e^{-\dfrac{8t^2}{\Dt^2}}\tag{Approximate $\delta$}
% \end{align*}

% DEFINE LOTS OF R...
% <<figcode,echo=FALSE>>=
% periodT <- 100
% rhoT <- 0.2
% Mp <- 100
% resolution <- 0.0001
% lambda <- 0.1
% deltaT <- periodT * rhoT
% MpDt <- Mp / deltaT
% barMpDrho <- Mp / (rhoT * periodT)
% xlim <- c(-periodT/2,periodT/2)
% xtime <- seq(xlim[1],xlim[2],periodT*resolution)
% MStepPeak <- Mp / (lambda*deltaT) * (1 - exp(-lambda*deltaT)) / (1 - exp(-lambda*periodT))
% 
% Tlambda <- periodT*lambda
% taulim <- c(-1/2,1/2)
% xtau <- seq(taulim[1],taulim[2],resolution)
% D <- 1 / (exp(Tlambda) - 1)
% library(bvpSolve)
% Msol<-function(E) {
%   bvpshoot(
%     yini=c(y1=NA),
%     x=xtime,
%     func=function(t,y,parms) {
%       list(c(E(t) - lambda*y[1]))
%     },
%     guess=c(Mp/periodT),
%     yend=function(Y,yini,parms) {
%       c(Y[1] - yini[1])
%     }, 
%     parms=c(E=E,lambda=lambda),
%     maxiter=1e6
%   )
% }
% Mplot <- function(E,ylim=c(0,1.5*MStepPeak)) {
%   sol <- Msol(E)
%   plot(sol[,1],sol[,2],
%        type="l",
%        xlim=xlim,
%        ylim=ylim,
%        ylab="", xlab="",
%        xaxs="i", yaxs="i",
%        xaxp=c(-periodT/2,periodT/2,10),
%        bty="n")
% }
% MplotAdder<-function(Eref) {
%   refSol <- Msol(Eref)
%   function(E) {
%     plot(refSol[,1],refSol[,2],
%          type="l",
%          xlim=c(-periodT/2,periodT/2),
%          ylim=c(0,1.5*MStepPeak),
%          ylab="", xlab="",
%          xaxt="n", yaxt="n", lwd=2,
%          #xaxp=c(-periodT/2,periodT/2,10),
%          bty="n",col="lightgray")
%     axis(2,seq(0,1.5*MStepPeak,.5*MStepPeak),FALSE)
%     axis(3,seq(-periodT/2,periodT/2,10),FALSE)
%     sol <- Msol(E)
%     lines(sol[,1],sol[,2],lwd=2)
%   }
% }
% EplotAdder<-function(Eref) {
%   ytime <- Eref(xtime)
%   function(E) {
%     plot(xtime,ytime,
%          type="S",
%          xlim=c(-periodT/2,periodT/2),
%          ylim=c(0,2*MpDt),
%          ylab="", xlab="",
%          xaxt="n", yaxt="n", lwd=2,
%          #xaxp=c(-periodT/2,periodT/2,10),
%          bty="n",col="lightgray")
%     axis(2,seq(0,2*MpDt,.5*MpDt),FALSE)
%     axis(1,seq(-periodT/2,periodT/2,10),labels=FALSE)
%     lines(xtime,E(xtime),lwd=2)
%   }
% }
% # E function factories
% stopper<-function(which, deltaT=1,MpDt=1,rho=0,periodT=1,Mp=1) {
%   if (deltaT <= 0 | MpDt <= 0 | rho < 0 | rho > 1 | periodT <= 0 | Mp <= 0)
%     stop(paste(which,": check arguments",sep=" "))
% }
% # step and modified step functions
% EstepFac <- function(deltaT, MpDt, rho=0) {
%   stopper("EstepFac", deltaT=deltaT, MpDt=MpDt, rho=rho)
%   b <- deltaT/2
%   if (rho == 0) { # step function
%     function(t) {
%       ifelse(abs(t) <= b, MpDt, 0)
%     }
%   } else {
%     base <- 2*MpDt / (2-rho)
%     sloped<-function(abst) { (base/rho)*(1-abst/b) }
%     if (rho != 1) { # trapezoid
%       function(t) {
%         abst<-abs(t)
%         ifelse( abst<=b,
%                 ifelse( abst<=b*(1-rho),
%                         base,
%                         sloped(abst)
%                 )
%           ,0)
%       }
%     } else { # triangle
%       function(t) { 
%         abst<-abs(t)
%         ifelse( abst<=b, sloped(abst) ,0)
%       }
%     }
%   }
% }
% 
% # delta function
% EdeltaFac <- function(deltaT, MpDt) {
%   stopper("EdeltaFac", deltaT=deltaT, MpDt=MpDt)
%   C <- MpDt * 2 * sqrt( 2/pi )
%   var <- -8 / deltaT^2
%   function(t) {
%     C * exp(var * t^2)
%   }
% } 
% 
% # delta function composed with trigonometric function
% EdeltaTrigFac <- function(deltaT, MpDt, periodT) {
%   stopper("EdeltaTrigFac", deltaT=deltaT, MpDt=MpDt, periodT=periodT)
%   w <- pi / periodT
%   C <- MpDt * 2*sqrt( 2/pi )
%   var <- -8 / (w*deltaT)^2
%   function(t) {
%     C * exp(var * sin(w*t)^2)
%   }
% }
% 
% # conventional trig-based function for seasonality
% EtrigFac <- function(Mp, periodT) { 
%   stopper("EtrigFac",Mp=Mp, periodT=periodT)
%   C <- Mp / periodT
%   w <- 2 * pi / periodT
%   function(t) {
%     C *(1+cos(w*t))
%   }
% } 
% 
% # trig-based function w/ parameter to squeeze peak
% EtrigPropFac <- function(Mp, deltaT, periodT) {
%   stopper("EtrigPropFac",Mp=Mp, deltaT=deltaT, periodT=periodT)
%   omega <- pi / periodT
%   n <- floor((2*sin( pi*deltaT / (periodT*2) )^2)^(-1))
%   topf <- factorial(n)
%   bf <- factorial(2*n)
%   C <- 2^(2*n)*topf^2 / (periodT*bf) * Mp
%   function(t) {
%     C*cos(omega * t)^(2*n)
%   }
% } 
% 
% # double logistic
% E2logFac<-function(Mp,deltaT,c) {
%   stopper("E2logFac", Mp=Mp, deltaT=deltaT)
%   C <- Mp / deltaT 
%   cdt <- c*deltaT / 2
%   function(t) {
%     C * ((1+exp(-c*t - cdt))^(-1) - (1+exp(-c*t + cdt))^(-1))
%   }
% }
% Estep <- EstepFac(deltaT,MpDt); # check("estep",Estep)
% Etrap <- EstepFac(deltaT, MpDt, .5); # check("etrap",Etrap)
% Etri <- EstepFac(deltaT, MpDt, 1); # check("etri",Etri)
% Edelta <- EdeltaFac(deltaT, MpDt); # check("edelta",Edelta)
% Edeltatrig <- EdeltaTrigFac(deltaT, MpDt, periodT); # check("edeltatrig",Edeltatrig)
% Etrig <- EtrigFac(Mp,periodT); # check("etrig",Etrig)
% Etrigprop <- EtrigPropFac(Mp,deltaT,periodT); # check("etrigprop",Etrigprop)
% E2log<-E2logFac(Mp,deltaT,1); # check("e2log",E2log)
% EtAdder<-EplotAdder(Estep)
% MpAdder<-MplotAdder(Estep)
% PlotStacker<-function(E,name) {
%   par(mfcol=c(2,1),mar=c(2,2,1,1))
%   EtAdder(E)
%   MpAdder(E)
% }
% @

% THEN USE IT SEVERAL TIMES
% \section{THINGS}
% \subsection{Modified Step (Trapezoid)}
% \begin{figure}
% \begin{center}
% <<pstrap,echo=FALSE,fig=TRUE>>=
% PlotStacker(Etrap)
% @
% \end{center}
% \end{figure}
% 
% \subsection{Modified Step (Triangle)}
% \begin{figure}
% \begin{center}
% <<pstri,echo=FALSE,fig=TRUE>>=
% PlotStacker(Etri)
% @
% \end{center}
% \end{figure}
% 
% \subsection{Approx. $\delta$}
% \begin{figure}
% \begin{center}
% <<psdel,echo=FALSE,fig=TRUE>>=
% PlotStacker(Edelta)
% @
% \end{center}
% \end{figure}

% \section{BIBLIOGRAPHY EXAMPLE (CITE ON PREV SLIDE)}
% \bibliography{biblio}{}
% \bibliographystyle{plain}

\end{document}
